% !TeX root = ../main.tex
% Add the above to each chapter to make compiling the PDF easier in some editors.


\section{Objectives and Contributions}

The thesis objective was to implement 



As part of this research, an attempt was made to directly integrate a coroutine based GPU scheduling system into an autonomous driving system. 
However, the lack of documentation and time as well as the complexity of the existing framework proved to be a limitation. 
As a result, this thesis pivoted to implementing a custom persistent thread scheduler on which coroutines can be implemented for real time systems.  
This approach enables long running GPU threads to receive new tasks, yield between them, and implement task prioritization in software. 
The software yielding emulates the desired preemption to improve responsiveness under load. 

The primary objective of this thesis is to explore GPU scheduling techniques that enable predictable, low latency execution for real time autonomous systems.  
In particular, the aim is to address the limitions of current GPU execution models, which prioritize throughput at the expense of timing guarantees, by investigating alternative scheduling strategies suitable for safety critical environments. 
Furthermore, this thesis aims to explore the GPU hardware architecture and CUDA programming model in order to design an efficient optimized scheduling strategy. 

The initial objective of this work was to integrate an existing coroutine based GPU scheduler into a autonomous driving framework.
This integration was intended to evaluate the viability of fine grained GPU scheduling within a complex, real time system. 
Additionally, this work further sought to evaluate the scheduling latencies in a coroutine based implementation, from which specific timing guarantees may be derived. 
Ultimately, given the steep learning curve of both the GPU coroutine framework and the autonomous driving platform, and my limited prior experience with GPU programming and compiler theory, the integration proved more difficult than expected and required a narrowing of scope.
As a result, this original objective was reconsidered.


The thesis therefore shifted focus to a more foundational and controlled apporach. 
Instead of embedding GPU scheduling into an existing system, this work focuses on designing and implementing a custom persistent thread scheduler using native CUDA. 
This scheduler serves as a minimal proof of concept foundation for enabling persistent GPU threads to receive tasks, yield between them, and emulate prioritization.
By building this system, it becomes possible to explore how real time behaviors can be emulated in software, to study the limitations of the CUDA execution model, and to understand teh design trade-offs involved in managing GPU concurrency manually.  

This revised objective emphasizes both the practical and conceptual aspects of real time GPU scheduling. 
Practically, it provides a working framework for testing scheduling behavior under load. 
Conceptually, it offers insights into the GPU architecture and CUDA programming model in maximizing the hardware capabilities and utilization.
The findings of this work aim to inform future efforts to integrate persistent thread or coroutine based scheduling into real world autonomous driving systems. 


	
In summary, this thesis investigates real time GPU scheduling techniques for autonomous driving.
By building a persistent thread architecture to support coroutine based task control, the goal is to reduce latency variability and enable timely execution of safety critical GPU workloads. 
This thesis aims to bridge the gap between the throughput oriented design of modern GPUs and the strict timing guarantees demanded by real time autonomous systems.

\section{Thesis Outline}
