% !TeX root = ../main.tex
% Add the above to each chapter to make compiling the PDF easier in some editors.

\chapter{Objectives}\label{chapter:Objectives}


The initial objective of this thesis was to implement a real time scheduling solution into an existing open source autonomous driving project. 
The open source driving project, Apollo\footnote{For more information, visit the GitHub repository: \url{https://github.com/ApolloAuto/apollo}}, is being developed by Baidu and currently supports a fully functional autonomous driving platform.
The system consist of both a \acs{CPU} and \acs{GPU} on the hardware side, with the CPU offloading compute heavy machine learning tasks to the \acs{GPU}.
The \acs{CPU} tasks run inside of CyberRT, built on a real time linux operating system, essential for maintaining real time deadlines. 
CyberRT already supports CPU side coroutines as a means of ensuring predictable system behavior, but does not support any GPU side fine grained scheduling. 
Instead of supporting the real time fine grained scheduling like the \acs{CPU}, the \acs{GPU} executes received tasks sequentially. 
This project means to further ensure the safety of passengers by furthering the fine grained scheduling capabilities of this system 

To improve the current Apollo \acs{GPU} scheduling, the real time \acs{GPU} scheduling solution LuisaCompute-Coroutine\footnote{For more information, visit the GitHub repository: \url{https://github.com/LuisaGroup/LuisaCompute-coroutine/tree/next}} was selected \cite{Zheng2022LuisaRender}.
This project provides an "automic solution [to] flexibly schedule and split rendering tasks" by implementing \acs{GPU} coroutines into Luisa. 
Luisa is a high-level GPU programming framework that facilitates the development of efficient and portable GPU code, making it well-suited for real-time applications like autonomous driving.
Integrating LuisaCompute’s coroutine based GPU scheduler into Apollo was expected to enable more flexible and efficient GPU task management, thereby improving the reliability and safety of the real time autonomous driving platform.

However, due to practical constraints, including complex integration dependencies between Apollo and LuisaCompute, the inherent complexity of both frameworks, limited documentation and support, and time constraints, the original objective proved infeasible within the scope of this thesis.
Consequently, the focus shifted to designing and implementing a custom persistent thread scheduler using native CUDA. 
This approach enabled a hands-on exploration of how persistent GPU threads operate, how they can be scheduled, and what challenges arise when managing long-lived GPU execution contexts manually. 
The revised objective became not only to prototype such a scheduler but to study its behavior, evaluate its limitations, and compare its implications with coroutine-based approaches like those in Luisa Compute.
This shift allowed for a deeper understanding of the underlying GPU execution model and provided a foundation for future work toward integrating persistent thread models into complex frameworks such as Apollo.






	
