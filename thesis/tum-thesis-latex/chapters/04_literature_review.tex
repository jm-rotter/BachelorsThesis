% !TeX root = ../main.tex
% Add the above to each chapter to make compiling the PDF easier in some editors.

\chapter{Literature Review}\label{chapter:Literature Review}

%\section{GPU Partitioning}

\section{GPU Coroutines for Flexible Splitting and Scheduling of Rendering Tasks}

Previous work, on which this thesis is based, offers a new method in rendering task management to demonstrate a flexible, fine-grained scheduling mechanism on \acsp{GPU}. 
Previously discussed in the background and introduction, this paper focuses on splitting tasks into discrete segments that can be suspended and resumed as needed on \acsp{GPU}. 
Specifically, large, monolithic kernels could be rewritten into smaller tasks using coroutines. 
This capability, important for tasks with downtimes, reduced the latency associated with waiting for an entire batch of rendering computation to complete and maximized the utilization of the \ac{GPU}'s parallel processing capabilities. 


In this paper, a flexible \ac{DSL} was presented, which allowed the writing in a mega-kernel fashion with \$suspend statements to define suspention marks, to define a coroutine.
These points, allow the creation of different discrete segments that can be suspended and resumed as needed. 
The \ac{DSL} has support for a multitude of languages, specifically C\# and CUDA. 
With minimal changes to the original code, the new DSL can be included, which becomes dynamicaly recorded and translated to an intermediate representation. 
From the intermediate representation, after a set of comiler transformation and analysis used to extract the state frame, a scheduler can define the execution and manage the state frames. 
Important here, is that the scheduler can be chosen or rewritten, which will be used for Apollo. 
Ultimately, through the design of an easy-to-use and easily accessible library, this work can be implemented into autonomous driving systems. 


