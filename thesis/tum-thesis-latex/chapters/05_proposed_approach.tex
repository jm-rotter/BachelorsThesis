% !TeX root = ../main.tex
% Add the above to each chapter to make compiling the PDF easier in some editors.

\chapter{Proposed Approach}\label{chapter:Proposed Approach}

\section{Section}



The proposed solution will use a library with an API to implement coroutines on persistent threads.
When a new task, from the CPU, is scheduled, the task must actively notify the GPU that a new task exists and the GPU needs to halt. 
This signaling will happen, by using GPU flags that can be set and evaluated or potentially through a maximum latency approach.
The maximum latency approach, will maintain a maximum latency between when a task is scheduled onto the GPU and when the task needs to start, by periodically checking for new tasks. 
Next a mechanism to saving the thread state and resetting the thread state will be implemented. 
The thread state consists of the \ac{PC}, thread register values, and memory state, which consists of thread private memory, warp and block level memory, and the persistant state. 
These values must be stored in the global memory and then resused to restore execution. 
Furthermore, to prevent deadlocks a synchronisation mechanism needs to be developed, to disallow deadlocks. 
If a GPU kernel uses a syncThreads() call and a coroutine attempts to yield it at that point, a deadlock will occur. 


