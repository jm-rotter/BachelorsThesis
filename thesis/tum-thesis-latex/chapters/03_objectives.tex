% !TeX root = ../main.tex
% Add the above to each chapter to make compiling the PDF easier in some editors.

\chapter{Objectives}\label{chapter:Objectives}

The objective for this thesis is to develop, implement and evaluate a persistant thread with coroutine based scheduling approach for GPU threads in Apollo\footnote{Apollo is an open-source project developed by Baidu. For more information, visit the GitHub repository: \url{https://github.com/ApolloAuto/apollo}}, an open source autonomous driving platform, to improve real-time safety and determinism by ensuring a predictable scheduling behavior. 
This requires analyzing the limitations of existing GPU scheduling techniques, which, due to their batch processing design, lack the ability to perform task switching. 
To adress this, the new coroutine based scheduling mechanism LuisaCompute-coroutine\footnote{\cite{Zheng2022LuisaRender} and the accompanying source code \url{https://github.com/LuisaGroup/LuisaCompute-coroutine/tree/next}} will be employed. 
The Luisa coroutine scheduling was designed for graphics rendering tasks, around which their \ac{DSL} is written. 
Therefore, the scheduler and its \ac{DSL} will need to be adapted to fit the requirements of the autonomous driving domain, which prioritizes responsiveness, fault tolerance, and strict timing guarantees.  
By integrating this feature into Apollo, the GPU will be able to use coroutines for more flexible and efficient task management, ultimately enhancing the reliability and safety of the real-time autonomous driving platform.  

\subsection{Measurement and Evalutation}

To measure the success of a coroutine based solution, the latency of new time sensitive task scheduling and deadline success will be evaluated in comparison to the old system.
Furthermore, the implementability and effectiveness of this application will be analyzed through practical experiments and benchmarks, focusing on how well the system maintains deterministic behavior under varying workloads and task complexity. 


	
