% !TeX root = ../main.tex
% Add the above to each chapter to make compiling the PDF easier in some editors.

\chapter{Objectives}\label{chapter:Objectives}

The objective for this thesis is to develop, implement and evaluate a persistant thread with coroutine based scheduling approach for GPU threads to improve real-time performance by ensuring a predictable scheduling behavior. 

This requires analyzing the limitations of existing GPU scheduling techniques, which, due to their batch processing design, lack the ability to perform task switching. 
To adress this, a new scheduling mechanism will be designed that allows \ac{GPU} execution to be halted by yielding, temporarily giving the processing resources to a different process. 
Resuming execution will involve saving and restoring the executing state for the previous process.
Enabling \ac{GPU}s to yield and switch between processes dynamically will improve responsiveness and reduce latency. 
The end completed solution will be a library that can be used with an API to automatically use coroutines on persistant threads.  

\subsection{Measurement and Evalutation}

To measure the success of a coroutine based solution, the latency of new time sensitive task scheduling and deadline success will be evaluated.
Unfortunately, this extra overhead, required to halt and restart cuda kernels, will result in overall weaker performance.
Furthermore, the success will be measured on the implementability and ease of use to programm with and combine into Apollo, an open source autonomous driving platform.


	
